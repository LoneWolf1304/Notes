\begin{appendices}
    \section{Equivalence Relations and Equivalence Classes}
    \section{Vector Space}
\section{Groups}
A group is a 
\section{Ring}
A structure $(R,+, \cdot)$ is a ring, with $R\neq \emptyset$ and $+$ and $\cdot$ two binary operations such that:
\begin{itemize}
    \item \textbf{Addition}: $(R,+)$ is an abelian group that is associative, has a zero element, an inverse and is commutative. 
    \item \textbf{Multiplication}: is associative.
    \item \textbf{Distribution: } $(a+b)\cdot c = a\cdot c + b\cdot c$ and $a\cdot (b+c) = a\cdot b + a\cdot c$
\end{itemize}
Well, since multiplication is not necessarily commutative in a ring, the distributive law is postulated as two laws.
\section{Module}
Let $R$ be a ring. Then a \textit{left R-module} is an abelian group $(M,+)$ together with a map $\cdot : R\times M \rightarrow M$ satisfying:
\begin{itemize}
    \item $\alpha \cdot (x+y) = \alpha \cdot x + \alpha \cdot y \ \forall \ \alpha \in R \text{ and } x,y\in M$
    \item $(\alpha+\beta)\cdot x = \alpha\cdot x + \beta\cdot y\ \forall \ \alpha,\beta \in R \text{ and } x\in M$
    \item $(\alpha \beta)\cdot x = \alpha \cdot (\beta \cdot x)$
    \item $\mathds{1}\cdot x = x$
\end{itemize}
The elements of the ring are sometimes called as \textit{scalars}. Since the scalars always appear on the left, it is called a left module. We can similarly define the right module but by module, we will always mean left module.
\section{Algebra}
An algebra over a field $K$ is a vector space $A$ with a multiplication map $\mu : A\times A \rightarrow A$ denoted by $\mu(a,b) = a\cdot b$, such that for all $a,b,c\in A$ and $r\in K$ we have: 
\begin{itemize}
    \item Associativity: $(a\cdot b)\cdot c = a\cdot (b\cdot c)$ 
    \item Right and Left Distributivity: $a\cdot (b+c) = a\cdot b + a \cdot c$ and $(a+b)\cdot c = a\cdot c + b \cdot c$
    \item Homogenity: $r(a\cdot b) = (ra)\cdot b = a\cdot (rb)$
\end{itemize}
The associativity property is not always required, though. If the binary operation is associative too, then we call it an \textit{associative algebra}, otherwise \textit{non-associative}.\\[0.2cm] The above conditions, apart from associativity, is equivalent of saying that the algebra satisfies the \textit{bilinearity property}, that is, for $m,n \in K$ and $a,b\in A$:
\begin{align*}
    \text{Linearity in first factor:} \qquad (m\ a+n\ b)\cdot c &= m\ a\cdot c +n\ b\cdot c \\
    \text{Linearity in second factor:}\qquad  c \cdot (m\ a+n\ b)  &= m \ c\cdot a +n \ c\cdot b
\end{align*}
If $A,A'$ are algebras over field $K$, then an algebra homeomorphism is a linear map $\mathcal{L}: A\rightarrow A'$ such that multiplication is preserved, that is, $\mathcal{L}(ab) = \mathcal{L}(a)\mathcal{L}(b) \ \forall \ a,b \in A$. 
\end{appendices}