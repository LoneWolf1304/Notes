\section{Metric Tensor: how yo mama's fatness is quantified!}
Let us consider the spherical polar coordinates $(r,\theta,\phi)$. Note that the coordinate displacement $d\phi$ does not have the dimension of length. So, while considering the displacement vector, we write $d\veb{r} \sim r\sin\theta d\phi \hat{\veb{\phi}}$. Thus, in general, for any displacement we write it in terms of the ``metric tensor'' $g^{ij}$ as:
$$ds^2 = g_{ij}x^ix^j$$
In rectangular coordinates, we have $g_{ij} = \delta_{ij}$, that is, the metric tensor is just the identity matrix. In cylindrical coordinates where $dx^1 = d\rho, dx^2 = d\phi, dx^3 = z$, we have:
$$g_{ij}= \begin{pmatrix}
    1 & 0 & 0\\
    0 & \rho^2 & 0\\
    0 & 0 & 1
\end{pmatrix}$$
Thus, the displacement is written as
$$ds^2 = g_{ij}dx^i dx^j = d\rho^2 + \rho^2 d\phi^2 + dz^2$$
In spherical polar coordinates, we have:
$$g_{ij} = \begin{pmatrix}
    1 & 0 & 0\\
    0 & r^2 & 0\\
    0 & 0 & r^2\sin^2\theta
\end{pmatrix}$$
Thus, the displacement is written as:
$$ds^2 = d\theta^2 + r^2 d\theta^2 + r^2\sin^2\theta d\phi^2$$
If all of $g_{ij}$ are non-negative we call that geometry ``Riemannian'' and if some of them are negative, it is termed ``pseudo-Riemannian''.\\[0.3cm]
\textbf{Now, how the hell do we calculate the components of the metric tensor?}\\[0.3cm]
Well we have previously seen how we could obtain the basis vectors using the partial derivatives of the coordinates. So, suppose we have to find the metric tensor for spherical polar coordinate system. For that, let us first write the position vector:
$$\veb{r} = x \veb{e}_x+y \veb{e}_y+z \veb{e}_z = r \sin\theta \cos\phi \ \veb{e}_x + r \sin\theta \sin\phi \ \veb{e}_y + r \cos\theta \ \veb{e}_z$$
From this we obtain:
\begin{align*}
    \veb{e}_r &= \pdv{\veb{r}}{r} =  \sin\theta \cos\phi \ \veb{e}_x + \sin\theta \sin\phi \ \veb{e}_y + \cos\theta \ \veb{e}_z\\
    \veb{e}_\theta &= \pdv{\veb{r}}{\theta} = r \cos\theta \cos\phi \ \veb{e}_x + r \cos\theta \sin\phi \ \veb{e}_y - r \sin\theta \ \veb{e}_z\\
    \veb{e}_\phi &= \pdv{\veb{r}}{\phi} = -r \sin\theta \sin\phi \ \veb{e}_x + r \sin\theta \cos\phi \ \veb{e}_y
\end{align*}
Now, we had defined the metric tensor components to be the scalar product of the basis vectors. Also note that since the basis vectors are orthogonal, there will be no cross terms, so the tensor is diagonal. Using this we have:
\begin{align*}
    g_{rr} &= \veb{e}_r\cdot \veb{e}_r = \sin^2\theta \cos^2\phi + \sin^2\theta \sin^2\phi + \cos^2\theta = 1\\
    g_{\theta\theta} &= \veb{e}_\theta\cdot \veb{e}_\theta = r^2 \cos^2\theta \cos^2\phi + r^2 \cos^2\theta \sin^2\phi + r^2 \sin^2\theta = r^2\\
    g_{\phi\phi} &= \veb{e}_\phi\cdot \veb{e}_\phi = r^2 \sin^2\theta \sin^2\phi + r^2 \sin^2\theta \cos^2\phi = r^2 \sin^2\theta
\end{align*}
This is exactly what we had written before. Thus using this procedure, we can easily find the components of the metric tensor and then just chill! 
\subsection{Metric in relativity:}
We now consider the case of Minkowski space, where the coordinate displacements between two events are described by four component vector (\textbf{4 vector}):
$$dx^\mu = (dt, dx,dy,dz)\equiv (dt, d\veb{r})$$
The spacetime interval can be written in terms of the metric tensor $g_{\mu\nu}$ as:
$$ds^2 = dt^2-dx^2-dy^2-dz^2= g_{\mu\nu} dx^\mu dx^\nu$$
Thus in this case the metric tensor is:
$$g_{\mu\nu} = \begin{pmatrix}
    1 & 0 & 0 & 0\\
     0 & -1 & 0 & 0\\
      0 & 0 & -1 & 0\\
       0 & 0 & 0 & -1\\
\end{pmatrix}$$
Note the convention. Other conventions include the $(+,-,-,-)$ or the obnoxious $(+,+,+,+)$ with an imaginary time coordinate $x^0=ict$. The overall thing is, spatial and temporal part should have some difference.\\[0.3cm] We also define the ``proper time'' as $d\tau = \frac{ds}{c}$. Since we take $c=1$, then both are equivalent but let's take $c$ to be $c$ for once. Then,
$$c^2 d\tau^2 = c^2dt^2-dx^2-dy^2-dz^2 = c^2dt^2\brac{1 - \frac{dx^2+dy^2+dz^2}{c^2dt^2}} = c^2dt^2\brac{1-\frac{v^2}{c^2}}$$
From this we have
$$\boxed{\dv{\tau}{t} = \sqrt{1- \frac{v^2}{c^2}}:=\frac{1}{\gamma}}$$

\subsection{Metric Tensor is a Tensor!}
Okay, we so since beginning we had been calling \textit{g} to be the metric \textit{tensor}. But hey, why so? Let us show that it is indeed a tensor. We have said that the displacement \sout{doesn't} shouldn't change. So we have:
$$ds^2 = ds'^2 \ \implies g_{\alpha \beta}x^\alpha x^\beta =  g'_{\mu \nu}x'^\mu x'^\nu$$
We already know how the coordinates transform, so well let's put that:
$$g_{\alpha \beta}x^\alpha x^\beta =  g'_{\mu\nu}\pdv{x'^\mu}{x^\alpha} \pdv{x'^\nu}{x^\beta} x^\alpha x^\beta$$
Since the coordinates are arbitrary (lol, the OG reason we use everytime to compare both sides), we have finally that:
$$g_{\alpha \beta} = \pdv{x'^\mu}{x^\alpha} \pdv{x'^\nu}{x^\beta}g'_{\mu\nu}$$
It indeed transforms like a tensor and hence is a tensor. 
\subsection{Relating Ordinary and Co/Contra components}
Okay, so for this section, forget the Einstein summation rule. We will explicitly use the Sigma \emoji{flexed-biceps} symbol to denote the sum when needed. \\[0.3cm]

So, we consider the `ordinary vectors' whose components are just the coefficients of the unit basis vectors. For example:
\begin{align*}
    \widetilde{\veb{A}} &= \widetilde{A}_x \hat{i}+\widetilde{A}_y \hat{j}+\widetilde{A}_z \hat{k}\\
    &=\widetilde{A}_\rho \hat{\rho}+\widetilde{A}_\phi \hat{\phi}+\widetilde{A}_z \hat{k}
\end{align*}
The subscripts do not mean contravariant here, these are just index of the components. To distinguish this, we use tilde to denote the ordinary vector components. Now, note that we had written the displacement as:
$$ds^2 = \sum\limits_{\mu\nu} g_{\mu\nu}dx^\mu dx^\nu$$
In a diagonal metric tensor then, each component of the metric can be written as $g_{\mu\nu} = h_\mu^2\delta_{\mu\nu}$ (no sum is there). For example, see the displacement in spherical polar coordinates:
$$ds^2 = dr^2 + r^2 d\theta^2 + r^2\sin\theta^2 d\phi^2$$
So, $h_1 = 1, h_2 = r, h_3 = r\sin\theta$. We then do this magic and define the contravariant component to be:
$$A^\mu = \frac{\widetilde{A}_\mu}{h_\mu}$$
\textbf{So why does that make sense?}\\[0.3cm]
Note that for ordinary vectors, we define the dot product as: 
$$\veb{A}\cdot\veb{B} = \widetilde{A}_1\widetilde{B}_1+\widetilde{A}_2\widetilde{B}_2+\widetilde{A}_3\widetilde{B}_3 =\sum\limits_\mu h_\mu^2 A^\mu B^\mu = \sum\limits_{\mu\nu} g_{\mu\nu}A^\mu B^\nu$$
We had used the definition of $h$ using the delta function and metric tensor. Thus we see that using this kind of definition, the dot product for ordinary vectors can be made analogous to the definition of the inner product using the metric tensor. Once we have the contravariant components, we can find the covariant using the metric tensor:
$$A_\mu = \sum\limits_\nu g_{\mu\nu}A^\nu = \sum\limits_\nu h^2_\mu \delta_{\mu\nu} A^\nu = h^2_\mu A^\mu = h_\mu \widetilde{A}^\mu$$
Let us see an example for the spherical polar coordinates:
\begin{alignat*}{3}
    A^1   & = \frac{\widetilde{A}_r}{1}        &\quad\quad  A_1   &= 1 \cdot \widetilde{A}_r \\
    A^2   & = \frac{\widetilde{A}_\theta}{r}   &\quad\quad  A_2   &= r \cdot \widetilde{A}_\theta \\
    A^3   & = \frac{1}{\sin\theta} \frac{\widetilde{A}_\phi}{r} &\quad\quad  A_3   &= r\sin\theta \cdot \widetilde{A}_\phi \\
\end{alignat*}
\subsubsection{Case of the Gradient}
We define the ordinary gradient in similar lines to the ordinary vectors. Given a function $f$, the ordinary gradient is just:
$$\nabla f \equiv \pdv{f}{x}\hat{i}+\pdv{f}{y}\hat{j}+\pdv{f}{z}\hat{k}$$
So basically we take derivatives with respect to the lengths $x,y \ \text{or} \ z$. On the other hand, we have `covariant' gradient $\partial_\mu$ where we take derivatives with respect to a coordinate. So, in systems where the coordinates do not have the unit of length, like $\theta$ does not have length unit, both of these will differ. We can relate them by:
$$\nabla_\mu = \frac{1}{h_\mu}\partial_\mu f$$
Here $\nabla_\mu$ means the ordinary derivative with respect to coordinate $\mu$ but with units of length. Let's clear this with an example of the dear spherical coordinates:
\begin{alignat*}{3}
    \nabla_r      & = \frac{1}{h_1} \partial_r f              &\quad& = \partial_r f \\
    \nabla_\theta & = \frac{1}{h_2} \partial_\theta f         &\quad& = \frac{1}{r} \partial_\theta f \\
    \nabla_\phi   & = \frac{1}{h_3} \partial_\phi f           &\quad& = \frac{1}{r \sin\theta} \partial_\phi f
\end{alignat*}
Note that using this, we easily get the formula for gradient in spherical coordinate:
$$\nabla f = \pdv{f}{r}\veb{\hat{r}} +\frac{1}{r} \pdv{f}{\theta} \veb{\hat{\theta}}+\frac{1}{r \sin\theta} \pdv{f}{\phi}\veb{\hat{\phi}}$$
\subsubsection{Jacobian}
We had often written (while doing some calculations in classical mechanics) $dm = \rho dV \equiv \rho dxdydz$. Now, since mass is a scalar (duh!), there is no reason to disbelief that $dm$ is also a scalar. We know that $dV$ is not a scalar (here by $dV$ I mean the product of the differentials), I mean $dV$ depends on the coordinate system, $dxdxydx \neq dx'dy'dz'$. And, since $\rho$ and $dV$ combine to give a scalar, $\rho$ can also not be a scalar. So, let's see how to deal with this.\\[0.3cm]
Let us take the example of a curve $C$ enclosing some area. We have two coordinate systems, $\{x,y\}$ and $\{x',y'\}$ with transformations given by:
$$x' = f(x,y)\quad\quad y'=g(x,y)$$ Let $a$ and $b$ be the point on the x-axis where the tangent to the curve is vertical. 
\begin{figure}[H]
    \centering
    

% Pattern Info
 
\tikzset{
pattern size/.store in=\mcSize, 
pattern size = 5pt,
pattern thickness/.store in=\mcThickness, 
pattern thickness = 0.3pt,
pattern radius/.store in=\mcRadius, 
pattern radius = 1pt}
\makeatletter
\pgfutil@ifundefined{pgf@pattern@name@_0n8ltlb4v}{
\pgfdeclarepatternformonly[\mcThickness,\mcSize]{_0n8ltlb4v}
{\pgfqpoint{0pt}{0pt}}
{\pgfpoint{\mcSize+\mcThickness}{\mcSize+\mcThickness}}
{\pgfpoint{\mcSize}{\mcSize}}
{
\pgfsetcolor{\tikz@pattern@color}
\pgfsetlinewidth{\mcThickness}
\pgfpathmoveto{\pgfqpoint{0pt}{0pt}}
\pgfpathlineto{\pgfpoint{\mcSize+\mcThickness}{\mcSize+\mcThickness}}
\pgfusepath{stroke}
}}
\makeatother

% Pattern Info
 
\tikzset{
pattern size/.store in=\mcSize, 
pattern size = 5pt,
pattern thickness/.store in=\mcThickness, 
pattern thickness = 0.3pt,
pattern radius/.store in=\mcRadius, 
pattern radius = 1pt}
\makeatletter
\pgfutil@ifundefined{pgf@pattern@name@_4mr61cj3i}{
\pgfdeclarepatternformonly[\mcThickness,\mcSize]{_4mr61cj3i}
{\pgfqpoint{0pt}{0pt}}
{\pgfpoint{\mcSize+\mcThickness}{\mcSize+\mcThickness}}
{\pgfpoint{\mcSize}{\mcSize}}
{
\pgfsetcolor{\tikz@pattern@color}
\pgfsetlinewidth{\mcThickness}
\pgfpathmoveto{\pgfqpoint{0pt}{0pt}}
\pgfpathlineto{\pgfpoint{\mcSize+\mcThickness}{\mcSize+\mcThickness}}
\pgfusepath{stroke}
}}
\makeatother
\tikzset{every picture/.style={line width=0.75pt}} %set default line width to 0.75pt        

\begin{tikzpicture}[x=0.75pt,y=0.75pt,yscale=-1,xscale=1]
%uncomment if require: \path (0,406); %set diagram left start at 0, and has height of 406

%Curve Lines [id:da3268185769184484] 
\draw [color={rgb, 255:red, 9; green, 115; blue, 231 }  ,draw opacity=1 ][pattern=_0n8ltlb4v,pattern size=6pt,pattern thickness=0.75pt,pattern radius=0pt, pattern color={rgb, 255:red, 0; green, 0; blue, 0}][line width=2.25] [line join = round][line cap = round]   (274.97,155.52) .. controls (274.37,145.59) and (280.27,138.05) .. (282.77,132.63) .. controls (293.46,109.5) and (303.74,105) .. (322.38,99.92) .. controls (331.47,97.45) and (348.94,98.88) .. (352.98,99.11) .. controls (367.28,99.9) and (385.23,117.54) .. (400.99,118.73) .. controls (416.31,119.89) and (430.73,112.02) .. (444.79,104.83) .. controls (450.12,102.11) and (456.39,98.04) .. (462.19,97.47) .. controls (481.39,95.6) and (500.22,106.92) .. (510.8,128.54) .. controls (514.31,135.72) and (518,142.7) .. (518,151.43) ;
%Curve Lines [id:da16328532122826311] 
\draw [pattern=_4mr61cj3i,pattern size=6pt,pattern thickness=0.75pt,pattern radius=0pt, pattern color={rgb, 255:red, 0; green, 0; blue, 0}][line width=2.25] [line join = round][line cap = round]   (274.97,155.52) .. controls (276.17,169.3) and (280.84,184.44) .. (287.57,194.76) .. controls (314.94,236.71) and (365.5,240.16) .. (402.79,225.83) .. controls (412.23,222.2) and (423.04,214.12) .. (431.59,207.84) .. controls (436.32,204.38) and (442.17,199.1) .. (447.79,198.85) .. controls (450.97,198.71) and (471.71,200.47) .. (477.2,199.67) .. controls (490.93,197.66) and (511.85,174.92) .. (515,162.06) .. controls (515.73,159.08) and (518,155.25) .. (518,151.43) ;

%Shape: Axis 2D [id:dp07891177298123697] 
\draw  (205,285.32) -- (563,285.32)(240.8,32) -- (240.8,313.47) (556,280.32) -- (563,285.32) -- (556,290.32) (235.8,39) -- (240.8,32) -- (245.8,39)  ;
%Shape: Axis 2D [id:dp3819773826851949] 
\draw [color={rgb, 255:red, 208; green, 2; blue, 27 }  ,draw opacity=1 ] (205.76,292.63) -- (556.2,219.49)(189.04,37.34) -- (246.55,312.87) (548.33,216.02) -- (556.2,219.49) -- (550.37,225.81) (185.58,45.22) -- (189.04,37.34) -- (195.37,43.17)  ;
%Straight Lines [id:da3916277214241677] 
\draw  [dash pattern={on 4.5pt off 4.5pt}]  (274.97,155.52) -- (274,285.47) ;
%Straight Lines [id:da014495688305964594] 
\draw  [dash pattern={on 4.5pt off 4.5pt}]  (518,151.43) -- (517,286.58) ;
\draw   (393.88,219.81) -- (409.36,222.83) -- (399.16,234.85) ;
\draw   (408.96,127.66) -- (396.17,118.43) -- (410.47,111.79) ;

% Text Node
\draw (566,276.4) node [anchor=north west][inner sep=0.75pt]    {$x$};
% Text Node
\draw (237,11.4) node [anchor=north west][inner sep=0.75pt]    {$y$};
% Text Node
\draw (563,207.4) node [anchor=north west][inner sep=0.75pt]  [color={rgb, 255:red, 208; green, 2; blue, 27 }  ,opacity=1 ]  {$x'$};
% Text Node
\draw (185,13.4) node [anchor=north west][inner sep=0.75pt]  [color={rgb, 255:red, 208; green, 2; blue, 27 }  ,opacity=1 ]  {$y'$};
% Text Node
\draw (271,285.4) node [anchor=north west][inner sep=0.75pt]    {$a$};
% Text Node
\draw (514,288.4) node [anchor=north west][inner sep=0.75pt]    {$b$};


\end{tikzpicture}

    \caption{Measuring area in two coordinate systems}
\end{figure}
\noindent
Now, the area enclosed by the curve in the $\{x,y\}$ system is given by:
$$A = \int\limits_a^b y_{\mathrm{blue}}(x) dx - \int\limits_a^b y_{\mathrm{black}}(x) dx = \int\limits_a^b y_{\mathrm{blue}}(x) dx + \int\limits_b^a y_{\mathrm{black}}(x) dx = -\oint\limits_{C}y dx$$ 
The negative sign comes since the curve is taken anti-clockwise. Similarly for the $\{x',y'\}$ system, we have:
\begin{align*}
    A' &= -\oint\limits_{C}y' dx' 
\end{align*}
Note that $y'$ in the integrand is the y' coordinate. We are basically integrating the value $ydx$ over the curve. Then we can write:
\begin{align*}
    A' &= -\oint\limits_{C}y' dx' \\
    &= -\oint\limits_{C}g(x,y)\brac{\pdv{f}{x}dx+\pdv{f}{y}dy}\\
    &= -\oint\limits_{C}(M_xdx + M_ydy)\\
    &= -\oint\limits_{C} \veb{M}\cdot d\veb{r}\\
&= -\int\int\limits_S (\nabla\times \veb{M})\cdot \veb{\hat{n}}\ da
\end{align*}
where have defined $M_x = g(x,y)\pdv{f}{x}\equiv gf_x$ and $M_y = g(x,y)\pdv{f}{y}\equiv gf_y$ and used Stokes' Theorem.
In our case, $\veb{\hat{n}}$ is along the z-direction and $da \equiv dxdy$, so only $(\nabla\times \veb{M})_z = \pdv{M_y}{x}-\pdv{M_x}{y}$ will survive. Thus, we finally have:
\begin{align*}
    A' &=-\iint\limits_S \brac{\pdv{(gf_y)}{x}-\pdv{(gf_x)}{y}}dxdy\\
    &=-\iint\limits_S \brac{\cancel{g\pdv{f_y}{x}}+f_y\pdv{g}{x}-\cancel{g\pdv{f_x}{y}}-f_x\pdv{g}{y}}dxdy\\
    &=-\iint\limits_S \brac{\pdv{f}{y}\pdv{g}{x} - \pdv{f}{x}\pdv{g}{y}}dxdy\\
    &=\iint\limits_S \brac{  \pdv{x'}{x}\pdv{y'}{y}-\pdv{x'}{y}\pdv{y'}{x}}dxdy\\
    &=\iint\limits_S \det \begin{bmatrix}
        \pdv{x'}{x} & \pdv{x'}{y} \\[0.2cm]
        \pdv{y'}{x} & \pdv{y'}{y}
    \end{bmatrix} dxdy
\end{align*}
Note that the terms in the second step cancelled because partial derivatives commute. Like we have terms like $\pdv{f_x}{y}\equiv \frac{\partial^2 f}{\partial x\partial y}$ and $\pdv{f_y}{x}\equiv \frac{\partial^2 f}{\partial y\partial x}$, both of which are equal. Now we define this weird determinant with a high-sounding name called Jacobian:
$$\mathrm{J} := \det \begin{bmatrix}
        \pdv{x'}{x} & \pdv{x'}{y} \\[0.2cm]
        \pdv{y'}{x} & \pdv{y'}{y}
    \end{bmatrix}$$
    Then the integral becomes:
    $$A' = \iint\limits_S  \mathrm{J}\ dxdy = \iint\limits_S dx'dy'$$
    In short notation, we write $\mathrm{J} = \left|\pdv{x'}{x}\right|$. Well the thing is, technically we should call the Jacobian matrix as $\mathrm{J}$ but it's fine I guess. Just understand from the context. So this serves two purposes: \begin{itemize}
        \item Firstly, it tells where the prime and the unprimed things come in the matrix. Since unprimed things are in the denominator, we have the unprimed in the denominator in the matrix also. 
        \item Secondly, it tells us where to put the Jacobian while doing the transformation. Like we can make the mathematicians crazy and write:
        $$\iint \left|\pdv{x'}{\cancel{x}}\right| \cancel{dxdy} \equiv \iint dx'dy'$$
        So the unprimed denominator cancels with the unprimed differentials and the net result is the primed thing. So we can say that the from primed to unprimed frame, the Jacobian always comes with `$dxdy$'
    \end{itemize}
    Generalising this to a volume element, we would have obtained $dx' dy' dz' = \mathrm{J}dx dy dz$
    \subsubsection{Relating Metric and Jacobian}
    We take the absolute value of the determinant of both sides of the metric tensor transformation equation:
    $$\left|\det g'_{\mu\nu}\right| = \left|\det\pdv{x'^\rho}{x^\mu}\right|\left|\det\pdv{x'^\sigma}{x^\nu}\right|\left|\det g_{\rho\sigma}\right|$$
    Note that the determinants written above are just representations, like they are actually determinants of the matrices whose components are written in the above expression. For simplicity, we will denote the absolute value of the determinant of something simply by $|\cdot|$. Thus, we have 
    $|g'| = \left|\pdv{x}{x'}\right|^2 |g|$
    We take the absolute value since we had seen that for pseudo-Riemannian metric, the determinant of $g$ is negative, like the Minkowski metric has determinant -1. Also note that since $\pdv{x}{x'}$ and $\pdv{x'}{x}$ are inverses of each other (again, their matrices), then we have
    $$\left|\pdv{x}{x'}\right| = \left|\pdv{x'}{x}\right|^{-1}$$ 
    Then we can write:
    $$\left|\pdv{x}{x'}\right||g| = \left|\pdv{x'}{x}\right||g'|\implies \mathrm{J}^{-1} |g| = J |g'|$$
We used the fact that determinant of the inverse matrix is just the inverse (reciprocal) of the determinant. Then we have 
$$\mathrm{J} = \sqrt{\frac{|g|}{|g'|}}$$
Then we can write the area transformation as:
$$\sqrt{|g'|}dx'dy' = \sqrt{|g|}dxdy$$
This quantity is a scalar and this is a \textit{scalar transformation}. Generalising this, we can write for arbitrary dimension:
$$\sqrt{|g'|}d^nx' = \sqrt{|g|}d^nx$$
So we saw how volume element changes between coordinate systems. Now, since we have to keep $dq$ or $dm$ for example, to be scalars then we need the density to transform as $\rho' = \mathrm{J}^{-1}\rho$, so that the jacobian cancels from the volume element and it gives us a scalar.\\[0.3cm]
\textbf{In a nutshell...}\\[0.3cm]
Any quantity that transform in the following way: 
$$Q' = J^w Q$$
is called a \textit{tensor density} of weight $w$. So the volume element is a tensor density of weight $+1$ and the charge or mass density is of weight $-1$. 