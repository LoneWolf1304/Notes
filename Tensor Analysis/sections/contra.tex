\section{Contravariant and Covariant: why the skibidi!}
Let us suppose we have a vector space $\mathbf{V}$ and two bases $\{\mathbf{e}_i\}$ and $\{\mathbf{e}'_i\}$. We can the write the transformation of the basis into one another as:
\begin{align*}
    \mathbf{e}_i &= \tensor*{\Lambda}{^j_i}\mathbf{e}'_j\\
    \mathbf{e}'_i &= \tensor*{(\Lambda^{-1})}{_i^j}\mathbf{e}_j
\end{align*}
Now if we have a vector, we can write it in terms of the basis vectors as:
$$\mathbf{x} =  (x')^j \mathbf{e}'_j = x^i \mathbf{e}_i = (x^i \tensor*{\Lambda}{^j_i}) \mathbf{e}'_j $$
From this we get: $(x')^j = \tensor*{\Lambda}{^j_i} x^i$. Well note that, in the transformation equation of the basis, if we have the primed basis in the left, then we had the inverse transformation matrix $\Lambda^{-1}$ in the right, but here it is different (primed component in the left and $\Lambda$ in the right). Thus, the basis vectors and the components transform in the ``opposite'' or ``\textbf{contra}ry'' way. Thus, these components are called the \textbf{contravariant} components of the vector.\\[0.3cm]
Let us now consider the dual space $\mathbf{V}^*$\footnote{The dual space is the set of all linear functionals, that is, linear maps $f: \mathbf{V} \to \mathbb{R}$. One example is say the \textit{bra} vector which is dual to the \textit{ket}. So basically a bra takes a ket and returns a real (complex) number: $\braket{\psi|\psi}$ (braket)} of the vector space $\mathbf{V}$. From the linearity property, we have:
$$f(x^i\mathbf{e}_i) = x^i f(\mathbf{e}_i) \equiv x^i f_i$$
Now, we use the basis transformation equation:
$$f_i = f(\mathbf{e}_i) = f(\tensor*{\Lambda}{^j_i}\mathbf{e}'_j) = \tensor*{\Lambda}{^j_i}f(\mathbf{e}'_j) = \tensor*{\Lambda}{^j_i}f'_j$$
These $f_i$ are the components of the ``dual vector''. Note that if we have unprimed things on the left, then we have the transformation matrix $\Lambda$ on the right, which is similar to the transformation of the basis. Thus, we see that this transformation follows the same transformation as the basis vectors. Thus, these components are called the \textbf{covariant} components of the vector.\\[0.3cm]
So, the components are named according to how the basis vectors transform. If they transform together, the are called \textbf{co}variant (and denoted by downstairs index) and if they transform in the opposite way, they are called \textbf{contra}variant (and denoted by upstairs index). The contravariant and the covariant components together form an `invariance' like the scalar product (which do not change under coordinate transformation):
$$\veb{v}\cdot\veb{v} = v_i v^i $$
\noindent
We had earlier seen another definition of the inner product, using both contravariant components and the matric tensor, which was $\veb{v}\cdot\veb{v} = v^i v^j g_{ij}$. Comparing both these definitions, we can see a relation:
$$v_i = g_{ij} v^j$$
Thus when changing from contravariant to covariant, we just need to invoke the holy metric tensor (to be dicussed later further).\\[0.3cm]
\noindent
\textbf{Note:} In the Cartesian coordinates, the metric tensor is the Kronecker delta, that is, $g_{nm} = \delta_{nm}$ and hence the components of the vectors and dual vectors are the same, that is, $x^i = x_i$. 